\let\negmedspace\undefined
\let\negthickspace\undefined
\documentclass[journal,12pt,twocolumn]{IEEEtran}
\usepackage{cite}
\usepackage{amsmath,amssymb,amsfonts,amsthm}
\usepackage{algorithmic}
\usepackage{graphicx}
\usepackage{textcomp}
\usepackage{xcolor}
\usepackage{txfonts}
\usepackage{listings}
\usepackage{enumitem}
\usepackage{mathtools}
\usepackage{gensymb}
\usepackage{comment}
\usepackage[breaklinks=true]{hyperref}
\usepackage{tkz-euclide} 
\usepackage{listings}
\usepackage{gvv}                                        
\def\inputGnumericTable{}                                 
\usepackage[latin1]{inputenc}                                
\usepackage{color}                                            
\usepackage{array}                                            
\usepackage{longtable}                                       
\usepackage{calc}                                             
\usepackage{multirow}                                         
\usepackage{hhline}                                           
\usepackage{ifthen}                                           
\usepackage{lscape}
\setlength{\arrayrulewidth}{0.5mm}
\setlength{\tabcolsep}{18pt}
\renewcommand{\arraystretch}{1.5}
\newtheorem{theorem}{Theorem}[section]
\newtheorem{problem}{Problem}
\newtheorem{proposition}{Proposition}[section]
\newtheorem{lemma}{Lemma}[section]
\newtheorem{corollary}[theorem]{Corollary}
\newtheorem{example}{Example}[section]
\newtheorem{definition}[problem]{Definition}
\newcommand{\BEQA}{\begin{eqnarray}}
\newcommand{\EEQA}{\end{eqnarray}}
\newcommand{\define}{\stackrel{\triangle}{=}}
\theoremstyle{remark}
\newtheorem{rem}{Remark}
\begin{document}
\title{Waves(20) 11.15}
\author{EE23BTECH11051-Rajnil Malviya}
\date{January 2024}
\maketitle
\subsection*{\textit{Question :-}}
A train, standing at the outer signal of a railway station blows a whistle of frequency
400 Hz in still air. (i) What is the frequency of the whistle for a platform observer
when the train (a) approaches the platform with a speed of $10 ms^{-1} $, (b) recedes
from the platform with a speed of $10 ms^{-1} $? (ii) What is the speed of sound in each
case ? The speed of sound in still air can be taken as $340 ms^{-1} $.
\bigskip
 This problem requires knowledge of \textit{Doppler Effect} , So first we will learn Doppler effect and then we will solve our problem . Before learning Doppler effect , we will also understand Sound Waves .
 \begin{table}[h!]
   
        \begin{tabular}{ | m{1.0cm} | m{4cm} | } 
  \hline
 Symbol & Description \\ 
 \hline
 $y(t)$ & instantaneous displacement of wave \\
 \hline
$f $& frequency transmitted by source \\
\hline
$f_r $& received frequency \\
\hline
$A$ & amplitude of wave \\ 
\hline
 k & wave number \\
\hline
$t $& time  \\
\hline
$\lambda $& wavelength of wave \\
\hline
$T $& time period of wave \\
\hline
\end{tabular}\\
\caption{}
\label{Table:1}
       
    \end{table}
\subsection*{Equation of Sound Wave :-}
Sound Wave is transmission of energy ; sound wave depends on many parameters . A general equation of sound wave is shown below 
\begin{equation} \label{eq1}
y(t) = A\sin\brak{2 \pi ft - kx}
\end{equation}

 \begin{table}[h!]
   
        \begin{tabular}{ | m{1.0cm} | m{3cm} |m{1.0cm}| } 
  \hline
 Symbol & Description  & Value\\ 
 \hline
 $f$ & frequency of source & 400Hz\\
\hline
$v$& velocity of sound in air   &  $340 ms^{-1} $\\
\hline
$v_o$& velocity of observer   &  $0 ms^{-1} $\\
\hline
$v_s$& velocity of source   &  $10 ms^{-1} $\\
\hline

\end{tabular}\\
\caption{}
\label{Table:2}
       
    \end{table}

From equation 1 , equation of sound wave when whistle is blown by
train is 
\newpage
$$y(t) = A\sin\brak{2 \pi \times400\times t - k x}$$ 
\;\;\;\;\;\;\;\;\;\;\;\;\;\;\;\;\;\;\;\;for this case $f\;=\;400Hz$\\
\subsection*{\textit{Physical interpretation of frequency :-}}
In context of sound waves , frequency represents number of cycles that occur in unit of time . In sound waves Higher frequency corresponds to high pitch and lower frequency corresponds to low pitch . Aging causes thinning of vocal cords , due this pitch become higher in old men and lower in old women .
\subsection*{\textit{Physical interpretation of Amplitude :-}}
In context of sound waves , amplitude is magnitude of sound wave . Higher the amplitude results in louder the sound and lower amplitude results in softer sound .
\subsection*{\textit{Physical interpretation of Wavelength :-}}
Wavelength of sound waves is distance between two consecutive compressions or rarefactions .
\subsection*{\textit{Wave number  :-}}
Wave number(k) is parameter used to represent number of wavelengths per unit distance .
\begin{align}k=\frac{2  \pi}{\lambda}\end{align}
\begin{table}[h!]
        \begin{tabular}{ | m{2.5cm} | m{2.5cm} | } 
  \hline
 Transmitted Signal & Received Signal\\
 \hline
 Source transmits & Source will receive  \\
  a signal with & reflection of its \\
   frequency &transmitted Signal\\
   \hline
    & \\
$A\sin{(2 \pi f t +\phi)}$&$A\sin{(2 \pi f_r t +\phi)} $ \\
    & \\
\hline
\end{tabular}\\
\caption{}
\label{Table:4}
    \end{table}

\newpage
\subsection*{\textit{Velocity of Sound in medium  :-}}
Velocity of sound (v) differs medium to medium , it influenced by some parameters . 
 \begin{table}[h!]
   
        \begin{tabular}{ | m{1.0cm} | m{4cm} | } 
  \hline
 Symbol & Description \\ 
 \hline
 $\gamma$ & adiabatic index or ratio of specific heats of medium  \\
 \hline
$\rho$& density of medium\\
\hline
$P $& pressure of medium \\
\hline
\end{tabular}\\
\caption{}
\label{Table:1}
       
    \end{table}
v in a particular medium is defined as \begin{align}v=\sqrt{\frac{\gamma  . P}{\rho}}\end{align}

\subsection*{\textit{Doppler Effect for Sound Waves :-}}
Doppler effect for sound wave refers to change in frequency or pitch of sound wave observed by an observer when there is a relative motion between observer and source .


\subsection*{\textit{General \;Equation \;of \;Doppler :-}}
Now we will see formulas for Doppler effect in different situations . \\
\begin{align}{\label{eq2}}f' = \frac{v'}{\lambda'}\end{align}
\begin{align}{\label{eq3}}f = \frac{v}{\lambda}\end{align}
\begin{align}{\label{eq4}}f' = \frac{v \pm v_o}{\lambda \pm v_s T}\end{align}
\begin{align}{\label{eq5}}f = \frac{1}{T}\end{align}
From equations 5 and 7 , then substituting in equation 6
\begin{align}{\label{eq6}}f' = \frac{\brak{v \pm v_o}f}{v \pm v_s }\end{align}
Above equation is general equation for any case in doppler effect .
Signs of equation 8 depends on velocities of both observer and source .
\subsection*{\textit{Effect \;of  \;observer's  \;velocity\;on \;frequency :-}}
If observer is moving towards source and source is stationary ($v_s=0)$, then sound will reach observer faster , so observed frequency will increase. It means \begin{align}{\label{eq7}}f'>f\end{align} From equation 8 \begin{align}{\label{eq8}} f' = \frac{\brak{v \pm v_o}f}{v}\end{align}\\
From equations 10 and 9 , 
\begin{align} \frac{\brak{v \pm v_o}f}{v} > f\end{align}
\begin{align}{\label{eq10}} v \pm v_o > v\end{align}
 There must be (+) sign to satisfy equation 12 .
And for vice-versa case(observer is moving away) , so frequency will be less than $f_n$
So ,\begin{align}{\label{eq11}}f'<f\end{align}
From equation 10 ,
\begin{align}\frac{\brak{v \pm v_o}f}{v} < f\end{align}
\begin{align}{\label{eq13}} v \pm v_o < v\end{align}
There must be (-) sign to satisfy equation 15 .
\subsection*{\textit{Effect \;of  \;source's  \;velocity\;on \;frequency :-}}
If source is moving towards stationary observer($v_o =0$), so $\lambda'$ in equation 4 will compress and denominator will decrease , so $f'$ will increase , \begin{align}f'>f\end{align}From equation  8
\begin{align}f' = \frac{vf}{v \pm v_s}\end{align}\\
From equations  10 and 5
\begin{align} \frac{v f}{v \pm v_s} > f\end{align}
\begin{align}{\label{eq17}} \frac{v}{v \pm v_s} > 1\end{align}
 There must be (-) sign to satisfy equation  19 .
And for vice-versa case(source is moving away) , so wavelength will increase and denominator in equation 6 increases so $f'$ will decrease \\
\begin{align}f'<f\end{align}
From equation 8
\begin{align}{\label{eq19}} f' = \frac{vf}{v \pm v_s}\end{align}\\
From equations 21 and 20 , 
\begin{align} \frac{v f}{v \pm v_s} < f\end{align}
\begin{align}{\label{eq21}} \frac{v}{v \pm v_s} < 1\end{align}
 There must be (+) sign to satisfy equation 23.\\\\
 $v_o$ and $v_s$ directly affect equation 8 , independent of each other . We can change signs of numerator and denominator independently by analysing situation .\\\\
    
Let's get back to our problem solution\\
(i).a When the train approaches the platform (i.e., the observer at rest),\\\\
\textit{Solution  :-}\\
So here source is approaching , wavelength will decrease and frequency will increase , so we have to increase $f' $ \\
\begin{align}{\label{eq22}}f' = \frac{v f}{v- v_s }\end{align}
On Substituting in equation 24
$$f'_a=400\brak{\frac{340}{340-10}}$$
$$f'=412.1212$$
\bigskip
b. When the train recedes the platform (i.e., the observer at rest), \\\\
\textit{Solution  :-}\\\\
It is vice-versa of above, From Table:3
\begin{align}f'=f\brak{\frac{v}{v+v_s}}\end{align}
$$f'=400\brak{\frac{340}{340+10}}$$
$$f'=388.5714$$\\
(ii) The speed of sound in each will be same.It is $340  ms^{-1}$ in each case.\\\\
From equation 8
we will interchange $v_o\; and\; v_s$ 
\begin{align}{\label{eq24}}f_r = f' \brak{\frac{{v \pm v_s}}{v \pm v_o }}\end{align}
For signs , we will use same logic, if someone is approaching another one , so definitely it will increase frequency($f_r$) and if receding so it will decrease frequency($f_r$) . \\\\
 In our problem  if , train approaches with $v_s10 ms^{-1} $ and if platform is taken as obstacle with $v_o=0$, \\
 Received frequency by platform is \textit{f'} 412.1212   , \\
 Here train(source) is approaching , so $v_s$ will try to increase $f_r$\\
On substituting in  equation 26
\begin{align}{\label{eq24}}f_r = f' \brak{\frac{{v \pm v_s}}{v }}\end{align}
 To increase $f_r$ , there must be (+) sign ;
 $$f_r = 412.1212\brak{\frac{340 + 10}{340 }}$$
 $f_r$=424.2424\\ \\So Received signal will be , \\\\
 From table Table:3
 $$y_r(t) = A\sin\brak{ 2 \pi  (424.2424)t -kx}$$
 if train is receding with $10 ms^{-1} $ , so $v_s$ will try to decrease $f_r$
 so we will use (-) sign in equation 26 with $v_o=0$\\
  $$f_r =\brak{ \frac{v -v_s}{v }}$$
  $f'$=388.5714 \\
  on substituting in equation  26
  $$f_r = 388.5714 \brak{\frac{340 -10}{340 }}$$
  $f_r=377.1428$
 \\ So received signal will be , \\
 From table Table:3
 $$y_r(t) = A\sin\brak{2 \pi  (377.1428)t -k x}$$
\end{document}
